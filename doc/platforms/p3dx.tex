%% Copyright (C) 2010, 2011, Gostai S.A.S.
%%
%% This software is provided "as is" without warranty of any kind,
%% either expressed or implied, including but not limited to the
%% implied warranties of fitness for a particular purpose.
%%
%% See the LICENSE file for more information.

\chapter{Pioneer 3-DX}
\label{sec:p3dx}
\setHtmlFileName{p3dx}

This UObject has been created to drive the Pioneer 3-DX robot. It uses the
Aria/Arnl manufacturer library.

\section{Getting started}

\subsection{Prerequisites}
\begin{itemize}
\item Have a Pioneer P3-DX robot with linux or any linux distribution with
  MobileSim installed (\url{http://robots.mobilerobots.com/wiki/MobileSim}).

\item Have Urbi Sdk and ARNL libraries installed
  (\url{http://robots.mobilerobots.com/wiki/ARNL,_SONARNL_and_MOGS}).
\end{itemize}

\subsection{Installation}
\begin{itemize}
\item Untar you archive file
\item Move in your archive and create a config.mk file where variables
  \env{UPATH} and \env{ARNLPATH} are defined. For example
\begin{shell}
UPATH=/usr/local/gostai	 # path for your urbi-sdk directory
ARNLPATH=/usr/local/Arnl # path for your arnl directory
\end{shell}
\item Run \command{make} to compile
\end{itemize}

\subsection{Running}
\begin{itemize}
\item If you are working with the simulator MobileSim, make sure it is
  running.
\item Run \samp{make run}.
\item Connect to the Urbi server, for instance with \samp{telnet IPYouNeed
    54000}.
\end{itemize}


\section{How to use Pioneer 3-DX robot}

The following objects are defined to support the Pioneer 3-DX Robot.  Please
note that when loaded, \file{p3dx.u} also defines the \lstinline{robot}
variable to enforce the conformance with the Gostai Standard Robotics API
(\autoref{sec:naming}).

Below, we denote read-only slots with \samp{r}, and read-write slots with
\samp{rw}.

\let\subsectionOrig\subsection
\let\subsection\subsectionObject

\subsection{P3dx}
\begin{urbiscriptapi}
\item[body] The \refObject{P3dx.body} class.


\item[go](<d>) Cover \var{d} meters.


\item[model] Model of the robot (r).


\item[moveTo](<list>) Try to plan a path to position \var{list} == [x,y,th]
  in absolute frame.  \code{mapFileName} must have a valid value and
  \code{motorsLoad} must be 1.


\item[name] Name of the robot (r).


\item[robotType] Type of the robot (r).


\item[serial] Serial number of the robot (r).


\item[stop] Stop any movement.


\item[turn](<r>) Turn \var{r} radians.
\end{urbiscriptapi}

\subsection{P3dx.body}

\begin{urbiscriptapi}
\item[load] Enable the motors (rw)


\item[safetyDistanceMax] Speed is not restrained if all obstacles


\item[safetyDistanceMin] Moving is not allowed if any obstacle is closer
  (rw)


\item[sonar] The object \refObject{P3dx.body.sonar}.


\item|wheel[left].speed| Left wheel translation speed (r)


\item|wheel[right].speed| Right wheel translation speed (r)
\end{urbiscriptapi}


\subsection{P3dx.body.odometry}
\begin{urbiscriptapi}
\item[coveredAngle] Total yaw angle covered (r)


\item[coveredDistance] Total distance covered (r)


\item[position] List for [x, y, z, yaw] in absolute frame (rw)


\item[x] Odometric x position (r)


\item[y] Odometric y position (r)


\item[yaw] Odometric yaw angle (r)


\item[z] Always 0 (r)
\end{urbiscriptapi}



\subsection{P3dx.body.sonar}

%% Don't want to push all this in the index, so don't use urbiscriptapi
%% although it's a natural choice.
\begin{itemize}
\item \lstinline|load|\\
  Enable the sonars (rw).
\item \lstinline|[front][left][left].val|\\
  value for the front/left/left sonar (r)
\item \lstinline|[front][left][front].val|\\
  value for the front/left/front sonar (r)
\item \lstinline|[front][front][left].val|\\
  value for the front/front/left sonar (r)
\item \lstinline|[front][front][right].val|\\
  value for the front/front/right sonar (r)
\item \lstinline|[front][right][front].val|\\
  value for the front/right/front sonar (r)
\item \lstinline|[front][right][right].val|\\
  value for the front/right/right sonar (r)
\item \lstinline|[right][front].val|\\
  value for the right/front sonar (r)
\item \lstinline|[right][back].val|\\
  value for the right/back sonar (r)
\item \lstinline|[back][right][right].val|\\
  value for the back/right/right sonar (r)
\item \lstinline|[back][right][back].val|\\
  value for the back/right/back sonar (r)
\item \lstinline|[back][front][right].val|\\
  value for the back/front/right sonar (r)
\item \lstinline|[back][front][left].val|\\
  value for the back/front/left sonar (r)
\item \lstinline|[back][left][back].val|\\
  value for the back/left/back sonar (r)
\item \lstinline|[back][left][left].val|\\
  value for the back/left/left sonar (r)
\item \lstinline|[left][back].val|\\
  value for the left/back sonar (r)
\item \lstinline|[left][front].val|\\
  value for the right/front sonar (r)
\end{itemize}

\subsection{P3dx.body.laser}

\begin{urbiscriptapi}
\item[angleMax] Angle of the first value in the robot frame (r)


\item[angleMin] Angle of the last value in the robot frame (r)


\item[laserDistanceMax] Maximum distance the laser can return (r)


\item[laserDistanceMin] Minimum distance the laser can return (r)


\item[lastCaptureTimestamp] Last time laser read its values (r)


\item[load] Connect to sick laser (rw)


\item[resolution] Angle between to laser values (r)


\item[val] 181 distances given by the laser (r)
\end{urbiscriptapi}


\subsection{P3dx.body.camera}

\begin{urbiscriptapi}
\item[load] Connect the camera mechanical aspect (rw)


\item[pitch] Pitch value for camera (rw)


\item[yaw] Yaw value for camera (rw)


\item[zoom] Zoom value for camera (rw)
\end{urbiscriptapi}

\subsection{P3dx.body.x}
\begin{urbiscriptapi}
\item[speed] Translation speed (rw)
\end{urbiscriptapi}

\subsection{P3dx.body.yaw}
\begin{urbiscriptapi}
\item[speed] Rotation speed (rw) are further (rw)
\end{urbiscriptapi}

\subsection{P3dx.planner}
\begin{urbiscriptapi}
\item[locFailureEvent] Event emit-ed if localization fails. In this case
  \code{mapFileName} is set to ``''.


\item[mapFileName] Path to a map file. Setting a string value will also try
  to start a localization task. \code{laserLoad} must be 1. This may fail
  if:
  \begin{itemize}
  \item the file is unknown;
  \item the robot cannot localize itself on the map (the initial position of
    the robot SHOULD be equal to the Home position inside the map file).
  \end{itemize}

  Value ``'' deletes the previous map and stops the localization task.

  At any time, it might be set to ``'' if the localization task fails.


\item[pathPlanningEvent] Event emitted when path planning ends (r) with first
  parameter equal to:
  \begin{sublist}
    \begin{description}
    \item[0] if the path planning fails;
    \item[1] if it succeeds;
    \item[2] if it is canceled for any reason (new path, manual motion
      commands, change of map).
    \end{description}
  \end{sublist}
\end{urbiscriptapi}

\subsection{P3dx.body.battery}
\begin{urbiscriptapi}
\item[voltage] Current voltage (r)
\end{urbiscriptapi}

\let\subsection\subsectionOrig


\section{Mobility modes}
Three modes are available in order to control the Pioneer 3-DX mobility:
\begin{itemize}
\item Use \code{go} and \code{turn} to choose a distance and an angle.
\item Use \code{speedDelta} and \code{speedAlpha} to choose translation and
  rotation speeds.
\item Use \code{moveTo} to select a goal on a map that must be reached thank
  to path planning. Note that \code{distMin} and \code{distMax} have no
  impact in this mode.
\end{itemize}

\section{About units}
Every physical quantity is in SI units (m, s, rad, ...).


%%% Local Variables:
%%% coding: utf-8
%%% mode: latex
%%% TeX-master: "../urbi-sdk"
%%% ispell-dictionary: "american"
%%% ispell-personal-dictionary: "../urbi.dict"
%%% fill-column: 76
%%% End:
