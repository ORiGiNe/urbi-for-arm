%% Copyright (C) 2011, 2012, Gostai S.A.S.
%%
%% This software is provided "as is" without warranty of any kind,
%% either expressed or implied, including but not limited to the
%% implied warranties of fitness for a particular purpose.
%%
%% See the LICENSE file for more information.

\chapter{ROS support}

\section{About ROS}

\href{http://www.ros.org}{ROS} (Robot  Operating System) is a set of tools and
libraries designed to provide a common framework for robotic application
developers. There are already thousands of compatible applications available in
various domains such as computer vision, navigation, visualization tools...

\section{ROS support in Jazz}

Jazz comes with complete ROS compatibility.
The currently used ROS version is 'electric'. A subset of the ROS distribution
is installed in the \lstinline|/opt/ros| directory of the Jazz hard drive.


\section{Enabling and configuring ROS service}

ROS support in Jazz is implemented as a Jazz service and is disabled
by default.  Refer to section \autoref{sec:jazz:profile} for
instructions on how to enable and disable services.

By default, a ROS master is started on the robot when the service is enabled.
You can specify a different ROS master by setting the
\lstinline|config.ros.master_uri| configuration variable:

\begin{urbiunchecked}
  config.ros.master_uri = "http://my_computer:11311";
  // Call _save to make the change persistent.
  config.ros._save;
\end{urbiunchecked}

\section{Predefined topics}

The Jazz ROS service maps each device of the robot to a ROS Topic with
similar name.  The following table lists the most important topics:

\begin{tabular}{llll}
  \hline
  Urbi name & ROS topic & Dir & ROS message type \\
  \hline
  \refObject{jazz.body.irs} & /robot/body/irs & Pub & sensor\_msgs/LaserScan \\
  \refObject{jazz.body.sonars} & /robot/body/sonars & Pub & sensor\_msgs/LaserScan \\
  \refObject{jazz.body.laser} & /robot/body/laser & Pub & sensor\_msgs/LaserScan \\
  \refObject{jazz.camera} & /robot/body/head/camera & Pub & sensor\_msgs/Image \\
  \multirow{2}{*}{\refSlot[jazz.body.head]{pitch}} & /robot/body/head/pitch\_out & Pub  & std\_msgs/Float32 \\
  &  /robot/body/head/pitch\_in & Sub & std\_msgs/Float32 \\
  \multirow{2}{*}{\refSlot[jazz.body.head]{yaw} } & /robot/body/head/yaw\_out & Pub & std\_msgs/Float32 \\
  &  /robot/body/head/yaw\_in & Sub & std\_msgs/Float32 \\
  \multirow{3}{*}{\refObject{jazz} } & /robot\_in & Sub & geometry\_msgs/Twist \\
    & /robot\_out & Pub &  nav\_msgs/Odometry \\
    & /tf & Pub & tf/tfMessage \\
  \hline
\end{tabular}


The sonar array and the laser if present are accessible as standard LaserScan
data. Robot odometry is available as either nav\_msgs/Odometry messages in the
/robot\_out topic or standard \href{http://www.ros.org/wiki/tf}{tf transforms}
on the /tf topic.

All sensor messages with a header are given in the \lstinline|base_footprint|
frame. tf messages from the odometry are published on the standard /tf topic,
performing the \lstinline|/base_footprint -> /odom| transform.

Head yaw and pitch are mapped to two different topics, since they have a
sensor part and an actuator part.

\section{Using the navigation stack with ROS}

If your Jazz is equipped with a laser scanner, Jazz is compatible with the
ROS \href{http://www.ros.org/wiki/navigation}{navigation stack}.

Navigation components can run on the Jazz CPU, or as remote components.

It is recommended to run them remotely, especially if you plan to use the
web control interface at the same time to avoid overloading the Jazz CPU.

\subsection{Creating a map of your environment}

The \href{http://www.ros.org/wiki/gmapping}{gmapping} module can be used to
create a map of your environment.

Make sure you are using the same ros-master as your Jazz. If you type
\lstinline|rostopic list| you should see all the topics advertised by the robot.

Then run:

\begin{shell}
rosrun gmapping slam_gmapping scan:=/robot/body/laser
\end{shell}

This will start gmapping. gmapping then processes odometry and laser scan data
to create a map of the environment as the robot moves.

Using the joystick (preferred) or the web interface, navigate the robot around
your environment. You should avoid fast rotations as it puts too much stress
on the SLAM process.

When you are done, extract the map using:

\begin{shell}
rosrun map_server map_saver
\end{shell}

This will create the file \lstinline|map.pgm| containing your map data.

You should then open the map in an image editor and crop it to remove the
empty space. This will reduce CPU requirements for the localization task.

\subsection{Autonomous navigation}

You will find on the robot a ROS package named \lstinline|jazz_navigation| under
\lstinline|JAZZ_ROOT/share/vigilant/ros| that
contains a reasonable default configuration for the navigation.

You can copy the whole directory to an external computer (recommended) or run
it on the robot:

\begin{shell}
# Copy the parameter files (replace myJazzName with the host name of your Jazz)
scp -r myJazzName:/usr/local/stow/vigilant/share/vigilant/ros jazz-ros
# Make the package visible to ros
export ROS_PACKAGE_PACK=$PWD/jazz-ros:$ROS_PACKAGE_PATH
# Run the navigation package
roslaunch jazz_navigation move_base.launch
\end{shell}

See the ROS
\href{http://www.ros.org/wiki/navigation/Tutorials/RobotSetup}{navigation tutorial}
for information about the various configuration files.

You can visualize sensor data and set navigation goals using the ROS
\href{http://www.ros.org/wiki/navigation/Tutorials/Using%20rviz%20with%20the%20Navigation%20Stack}{rviz}
tool:

\begin{shell}
rosrun rviz rviz
\end{shell}

Then load the \lstinline|jazz_navigation/nav.vcg| layout file in rviz, which
contains the proper rviz configuration for Jazz.

\subsection{Troubleshooting}

\subsubsection{Jazz does not move when setting a navigation Goal}

If Jazz is docked on its charging station, move commands are disabled. You must
remove Jazz from its charging station manually.

%%% Local Variables:
%%% coding: utf-8
%%% mode: latex
%%% TeX-master: t
%%% ispell-dictionary: "american"
%%% ispell-personal-dictionary: "../urbi.dict"
%%% fill-column: 76
%%% End:
