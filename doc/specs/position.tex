%% Copyright (C) 2010, 2011, Gostai S.A.S.
%%
%% This software is provided "as is" without warranty of any kind,
%% either expressed or implied, including but not limited to the
%% implied warranties of fitness for a particular purpose.
%%
%% See the LICENSE file for more information.

\section{Position}

This class is used to handle file locations with a line, column and file
name.

\subsection{Prototypes}
\begin{refObjects}
\item[Object]
\end{refObjects}

\subsection{Construction}

Without argument, a newly constructed Position has its fields initialized to
the first line and the first column.

\begin{urbiscript}[firstnumber=1]
Position.new;
[00000001] 1.1
\end{urbiscript}

With a position argument \var{p}, the newly constructed Position is a clone
of \var{p}.

\begin{urbiscript}
Position.new(Position.new(2, 3));
[00000001] 2.3
\end{urbiscript}

With two float arguments \var{l} and \var{c}, the newly constructed
Position has its line and column defined and an empty file name.

\begin{urbiscript}
Position.new(2, 3);
[00000001] 2.3
\end{urbiscript}

With three arguments \var{f}, \var{l} and \var{c}, the newly
constructed Position has its file name, line and column defined.

\begin{urbiscript}
Position.new("file.u", 2, 3);
[00000001] file.u:2.3
\end{urbiscript}

\subsection{Slots}

\begin{urbiscriptapi}
\item['+'](<n>)%
  Return a new Position which is shifted from \var{n} columns to the
  right.  The minimal value of the new position column is 1.
\begin{urbiassert}
Position.new(2, 3) + 2 == Position.new(2, 5);
Position.new(2, 3) + -4 == Position.new(2, 1);
\end{urbiassert}

\item['-'](<n>)%
  Return a new Position which is shifted from \var{n} columns to the
  left.  The minimal value of the new Position column is 1.
\begin{urbiassert}
Position.new(2, 3) - 1 == Position.new(2, 2);
Position.new(2, 3) - -4 == Position.new(2, 7);
\end{urbiassert}

\item['=='](<other>)%
  Compare the lines and columns of two Positions.
\begin{urbiassert}
Position.new(2, 3) == Position.new(2, 3);
Position.new("a.u", 2, 3) == Position.new("b.u", 2, 3);
Position.new(2, 3) != Position.new(2, 2);
\end{urbiassert}

\item['<'](<other>)%
  Order comparison of lines and columns.
\begin{urbiassert}
Position.new(2, 3) < Position.new(2, 4);
Position.new(2, 3) < Position.new(3, 1);
\end{urbiassert}

\item[asString]
  Present as \samp{\var{file}:\var{line}.\var{column}}, the file name is
  omitted if it is not defined.
\begin{urbiassert}
Position.new("file.u", 2, 3).asString == "file.u:2.3";
\end{urbiassert}

\item[column]
  Field which give access to the column number of the Position.
\begin{urbiassert}
Position.new(2, 3).column == 3;
\end{urbiassert}

\item[columns](<n>)%
  Identical to \lstinline|'+'(\var{n})|.
\begin{urbiassert}
Position.new(2, 3).columns(2) == Position.new(2, 5);
Position.new(2, 3).columns(-4) == Position.new(2, 1);
\end{urbiassert}

\item[file]
  The \lstinline|Path| of the Position file.
\begin{urbiassert}
Position.new("file.u", 2, 3).file == Path.new("file.u");
Position.new(2, 3).file == nil;
\end{urbiassert}

\item[line]
  Field which give access to the line number of the Position.
\begin{urbiassert}
Position.new(2, 3).line == 2;
\end{urbiassert}

\item[lines](<n>)%
  Add \var{n} lines and reset the column number to 1.
\begin{urbiassert}
Position.new(2, 3).lines(2) == Position.new(4, 1);
Position.new(2, 3).lines(-1) == Position.new(1, 1);
\end{urbiassert}

\end{urbiscriptapi}


%%% Local Variables:
%%% coding: utf-8
%%% mode: latex
%%% TeX-master: "../urbi-sdk"
%%% ispell-dictionary: "american"
%%% ispell-personal-dictionary: "../urbi.dict"
%%% fill-column: 76
%%% End:
