%% Copyright (C) 2011, Gostai S.A.S.
%%
%% This software is provided "as is" without warranty of any kind,
%% either expressed or implied, including but not limited to the
%% implied warranties of fitness for a particular purpose.
%%
%% See the LICENSE file for more information.

\section{Logger}

\lstinline|Logger| is used to report information to the final user or to the
developer. It allows to pretty print warnings, errors, debug messages or
simple logs. \lstinline|Logger| can also be used as \refObject{Tag} objects
for it to handle nested calls indentation. A log message is assigned a
category which is shown between brackets at beginning of lines, and a level
which defines the context in which it has to be shown (see
\autoref{sec:tools:env}). Log level definition and categories filtering can
be changed using environment variables defined in \autoref{sec:tools:env}.

\subsection{Examples}

The proper use of Loggers is to instantiate your own category, and then to
use the operator \lstinline|<<| for log messages, possibly qualified by the
proper level (in increase order of importance: \refSlot{dump},
\refSlot{debug}, \refSlot{trace}, \refSlot{log}, \refSlot{warn},
\refSlot{err}):

\begin{urbiscript}
var logger = Logger.new("Category")|;

logger.dump << "Low level debug message"|;
// Nothing displayed, unless the debug level is set to DUMP.

logger.warn << "something wrong happened, proceeding"|;
[       Category        ] something wrong happened, proceeding

logger.err << "something really bad happened!"|;
[       Category        ] something really bad happened!
\end{urbiscript}

You may also directly use the logger and passing arguments to these slots.

\begin{urbiscript}
Logger.log("message", "Category") |;
[       Category        ] message

Logger.log("message", "Category") :
{
  Logger.log("indented message", "SubCategory")
}|;
[       Category        ] message
[      SubCategory      ]   indented message
\end{urbiscript}

\subsection{Prototypes}
\begin{refObjects}
\item[Tag]
\end{refObjects}

\subsection{Construction}

\lstinline|Logger| can be used as is, without being cloned. It is possible
to clone \lstinline|Logger| defining a category and/or a level of debug.

\begin{urbiscript}
Logger.log("foo");
[        Logger         ] foo
[00004702] Logger<Logger>

Logger.log("foo", "Category") |;
[       Category        ] foo

var l = Logger.new("Category2");
[00004703] Logger<Category2>

l.log("foo")|;
[       Category2       ] foo

l.log("foo", "ForcedCategory") |;
[    ForcedCategory     ] foo
\end{urbiscript}

\subsection{Slots}

\begin{urbiscriptapi}

\item['<<'](<object>)%
  Allow to use the \lstinline|Logger| object as a \refObject{Channel}. This
  slot can only be used if both category and level have been defined when
  cloning.

\begin{urbiscript}
l = Logger.new("Category", Logger.Levels.Log);
[00090939] Logger<Category>
l << "foo";
[       Category        ] foo
[00091939] Logger<Category>
\end{urbiscript}

\item[debug](<message> = "", <category> = category)%
  Report a debug \var{message} of \var{category} to the user. It will be
  shown if the debug level is \lstinline|Debug| or \lstinline|Dump|. Return
  \this to allow chained operations.
\begin{urbiscript}
// None of these are displayed unless the current level is at least DEBUG.
logger.debug << "debug 1"|;
logger.debug("debug 2")|;
logger.debug("debug 3", "Category2")|;
\end{urbiscript}

\item[dump](<message> = "", <category> = category)%
  Report a debug \var{message} of \var{category} to the user. It will be
  shown if the debug level is \lstinline|Dump|. Return \this to allow
  chained operations.
\begin{urbiscript}
// None of these are displayed unless the current level is at least DEBUG.
logger.dump << "dump 1"|;
logger.dump("dump 2")|;
logger.dump("dump 3", "Category2")|;
\end{urbiscript}

\item[err](<message> = "", <category> = category)%
  Report an error \var{message} of \var{category} to the user. Return \this
  to allow chained operations.

\item[init](<category>)%
  Define the \var{category} of the new \lstinline|Logger| object. If no
  category is given the new \lstinline|Logger| will inherit the category of
  its prototype.

\item[Levels]%
  An \refObject{Enumeration} of the log levels defined in
  \autoref{sec:tools:env}.

\begin{urbiscript}
Logger.Levels.values;
[00000001] [None, Log, Trace, Debug, Dump]
\end{urbiscript}

\item[log](<message> = "", <category> = category)%
  Report a debug \var{message} of \var{category} to the user. It will be
  shown if debug is not disabled. Return \this to allow chained operations.
\begin{urbiscript}
logger.log << "log 1"|;
[       Category        ] log 1

logger.log("log 2")|;
[       Category        ] log 2

logger.log("log 3", "Category2")|;
[       Category2       ] log 3
\end{urbiscript}

\item[onEnter]%
  The primitive called when \lstinline|Logger| is used as a \lstinline|Tag|
  and is entered. This primitive only increments the indentation level.

\item[onLeave]%
  The primitive called when \lstinline|Logger| is used as a \lstinline|Tag|
  and is left. This primitive only decrements the indentation level.

\item[trace](<message> = "", <category> = category)%
  Report a debug \var{message} of \var{category} to the user. It will be
  shown if the debug level is \lstinline|Trace|, \lstinline|Debug| or
  \lstinline|Dump|. Return \this to allow chained operations.
\begin{urbiscript}
// None of these are displayed unless the current level is at least TRACE.
logger.trace << "trace 1"|;
logger.trace("trace 2")|;
logger.trace("trace 3", "Category2")|;
\end{urbiscript}

\item[warn](<message> = "", <category> = category)%
  Report a warning \var{message} of \var{category} to the user. Return \this
  to allow chained operations.
\begin{urbiscript}
logger.warn << "warn 1"|;
[       Category        ] warn 1

logger.warn("warn 2")|;
[       Category        ] warn 2

logger.warn("warn 3", "Category2")|;
[       Category2       ] warn 3
\end{urbiscript}
\end{urbiscriptapi}

%%% Local Variables:
%%% coding: utf-8
%%% mode: latex
%%% TeX-master: "../urbi-sdk"
%%% ispell-dictionary: "american"
%%% ispell-personal-dictionary: "../urbi.dict"
%%% fill-column: 76
%%% End:
